\documentclass{bcc}
\usepackage{amssymb}


\titulo{OrCA IDE: Uma IDE geradora de Ontologias}
\palavrasChave{Ontologia}{Linguagem Natural}{OWL}
\keywords{Ontology, Natural Language, OWL}

\autor{Paulo Mateus da Silva}{paulomatew@gmail.com}

\orientador{Ryan Azevedo}{UAG}{UFRPE}
%\orientadorDois{John von Neumann}{UAG}{UFRPE}
% \orientadorTres*{Ada Lovelace}{UAG}{UFRPE} % Se feminino use *, tanto para orientador ou examinador

\examinador{John Hopcroft}{UAG}{UFRPE}
\examinadorDois{Richard Karp}{UAG}{UFRPE}
\examinadorTres{Stephen Cook}{UAG}{UFRPE}
\examinadorQuatro{John Backus}{UAG}{UFRPE}

\dataMesAno{28}{maio}{2019}

\begin{document}

\selectlanguage{portuguese}

\capa

% \capaDois
\begin{agradecimentos}
Agradeço primeiramente e principalmente a Deus, que me sustentou durante todo este tempo em que estive neste curso, com sabedoria, saúde e principalmente humildade. Humildade essa que me ensinou meus pontos fortes e fracos, não só na área de TI e na área acadêmica, mas como pessoa no âmbito geral. 

Agradeço também aos meus pais, que embora no começo dessa jornada tenham duvidado da assertividade da minha decisão, estiveram ao meu lado nos momentos críticos para me dar ânimo e forças para querer continuar a luta por este, mais próximo do que nunca, diploma. Incontáveis vezes eu poderia citar ocasiões, durante o ensino fundamental e médio, que eles pediam, com vigor,  que eu deixasse o que estava fazendo para ir estudar. Não muito longe dali, na faculdade, incontáveis vezes ouvi deles mesmos para deixar de estudar e ir descansar. Não sei se é um sentimento desfuncional, mas isso funcionava como um combustível de alta propulsão. De igual forma agradeço aos meus irmãos, que me apoiaram desde o começo e que são pessoas que desfrutam do meu mais profundo carinho e admiração.

Agradeço também a minha esposa, que conheci neste curso, pelas incontáveis palavras de ajuda, incontáveis abraços e sorrisos de consolo, e imensa compreensão ao longo desse tempo juntos, como casados e como colegas de curso.

Agradeço a todos os professores que me fizerem crescer e me tornar uma pessoa, estudante e profissional melhor. Em especial, agradeço ao meu orientador e ao coordenador do curso por sua compreensão e ajuda nos momentos críticos e nos prazos apertados. Agradeço também ao stackoverflow, Google e YouTube, que são frutos da nossa amada tecnologia da informação, que me ampararam em tantos momentos de desespero e de pura curiosidade.

E por último mas não menos importante, agradeço a mim mesmo. Ao longo desses aproximadamente 6 anos e 7 meses de curso, não foi fácil passar tantas noites em claro programando, estudando, escrevendo e pesquisando... Não foi fácil manter a cabeça no lugar quando as reprovações e dificuldades vieram. Não foi fácil fazer este curso trabalhando 8h por dia, e por vezes ouvindo de tantos colegas que passaram o dia jogando cs ou dormindo. Mas dentre tantas leis naturais, essa é uma das que acredito piamente, a lei do retorno. Essa pode até tardar, mas nunca falha. Quem planta, colhe. Acabo este curso com a cabeça erguida, com o sentimento de dever cumprido e disposto a um novo mundo de possibilidades. Este curso mudou a minha vida e a vida da minha família. Agradeço a todas as pessoas que fizeram isso ser possível. Não tenho dúvidas que passarei este legado a frente.

Neste trabalho, encerro meu vínculo com este curso, mas creio estar pra sempre ligado a esta universidade. 
\end{agradecimentos}


\begin{resumo}
INSERIR RESUMO
\end{resumo}

\selectlanguage{english}
\begin{abstract}
INSERIR ABSTRACT
\end{abstract}
\selectlanguage{portuguese}

% Centralizar titulos
\renewcommand\contentsname{\centerline{Sumário}}
\renewcommand\listfigurename{\centerline{Lista de Figuras}}
\renewcommand\listtablename{\centerline{Lista de Tabelas}}

\lhead{Sumário}
\tableofcontents

\listoffigures
\addcontentsline{toc}{chapter}{Lista de Figuras}

\listoftables
\addcontentsline{toc}{chapter}{Lista de Tabelas}

\inicio
\chapter{Introdução}

TEXTO


\section {Contextualização}

O campo do Processamento de Linguagem Natural (PLN) visa converter a linguagem humana em uma representação formal que é fácil para os computadores manipularem. As aplicações finais atuais incluem a extração de informações, tradução automática, resumo, pesquisa e interfaces de computadores humanos \cite{collobert2008}.

O objetivo do PLN é construir modelos computacionais de linguagem natural para sua análise e desenvolvimento. Há uma motivação tecnológica para a construção de sistemas de computadores inteligentes como sistema de tradução automática, interfaces de linguagem natural para bancos de dados, interfaces homem-máquina para computadores em geral, sistemas de compreensão de fala, sistemas de análise e compreensão de textos, sistema de instrução assistido por computador, sistemas que leem e entendem texto impresso ou manuscrito \cite{bharati1995}.

Ontologias estão sendo reconhecidas como componentes importantes dos sistemas de informação e processamento de informações \cite{estival2004}. Os desenvolvedores de ontologia geralmente capturam os relacionamentos entre entidades como relacionamentos formais e lógicos. Para fazer isso, geralmente se utiliza a lógica descritiva \cite{liu2011}.


\section{Definição do Problema}

Em seu trabalho \cite{estival2004} fornece algumas sugestões sobre como o processamento de linguagem natural pode se beneficiar do uso de ontologias. É apresentado um projeto de pesquisa em larga escala no qual ontologias são parte do processo de raciocínio utilizado para o gerenciamento de informações e para a apresentação de informações. No trabalho, o acesso dos usuários à informação e a apresentação de informações aos usuários são mediados via linguagem natural, e as ontologias utilizadas no componente de raciocínio são acopladas ao léxico utilizado no componente de linguagem natural.

Caracterização do Problema:
Por que o Problema é um problema:
Algumas das Soluções Existentes e por que não são tão adequadas a solução do nosso problema:
Solução Adotada (Nossa Solução):
Motivação para a Solução Adotada:
Resumo dos Resultados Obtidos nos experimentos:

\section{Motivação e Justificativa}

As ontologias estabelecem um nível de abstração superior, não ambíguo e comum para vários domínios do conhecimento \cite{rocha2014}. Não há dúvida de que, em todos os casos, as ontologias oferecem uma maneira de abordar o significado dos termos (conceitos, relações) necessários para o processamento da informação \cite{estival2004}.


\section{Objetivos}

A seguir são descritos os objetivos gerais e específicos que foram definidos e que este trabalho deseja atingir.

\section{Objetivo Geral}
O objetivo geral deste trabalho foi desenvolver uma ferramenta computacional baseada em Processamento de Linguagem Natural para construção automática de Ontologias a partir de texto (Linguagem Natural Controlada). A esse objetivo Geral apresentamos os seguintes objetivos específicos.

\subsection{Objetivos Específicos}
•	Desenvolvimento de uma Linguagem Natural Controlada para criação de código OWL DL;

•	Construir ontologias em OWL DL com expressividade de alto nível (mínima ALC);

•	Apoiar engenheiros e desenvolvedores no processo de construção de bases de conhecimento modeladas como ontologias;

•	Permitir que usuários acelerem o desenvolvimento das bases de conhecimento modeladas como ontologias no qual são gerados automaticamente códigos OWL DL a partir dos textos inseridos;

•	Construir uma ferramenta capaz de realizar raciocínio de subsunção deduzindo que classes são subclasses de outras, a partir de suas respectivas descrições;

•	Detectar e verificar inconsistências em tempo de desenvolvimento nas ontologias construídas.


\section{Escopo e Limitações}

\section{Organização do Texto}

Além deste capítulo inicial que traz uma introdução sobre o tema, a motivação e justificativa para realização do trabalho, a definição do problema e o objetivo; este trabalho está organizado em mais 4 capítulos, como seguem:

• O \autoref{chap:fundamentacao} apresenta a fundamentação teórica, contendo referências bibliográficas de estudos na área e das ferramentas e conceitos que foram utilizados no desenvolvimento;

• O \autoref{chap:sistema} mostra a arquitetura do sistema e a maneira que ele foi construído, utilizando os conceitos e ferramentas que foram previamente apresentados no capítulo \autoref{chap:fundamentacao}.

• O \autoref{chap:exp} expõe os experimentos que foram realizados e seus respectivos resultados, exibindo assim, o funcionamento do sistema.


\chapter{Fundamentação teórica}
\label{chap:fundamentacao}

Nesta seção são apresentados os conceitos e ferramentas que foram utilizados no desenvolvimento deste projeto e as respectivas justificativas que embasam o uso de cada um deles.

\section{Ontologia}

Existem diversas técnicas que podem ser utilizadas para organizar informações e conhecimento, a aplicação de ontologias tem recebido uma atenção cada vez maior \cite{almeida2014}. O termo pode ser encontrado não apenas na ciência da computação, mas também em diversas áreas do conhecimento, por isso, é importante entender seu significado e sua aplicação em cada área específica.

O termo ontologia teve origem no século XVII na área da filosofia, a palavra vem do grego e pode ser traduzida como estudo da existência \cite{guizzardi2005}. Segundo \cite{gruber1995} o termo é emprestado da filosofia onde uma Ontologia é uma explicação sistemática da existência. Para um sistema de Inteligência Artificial, o que existe é tudo aquilo que pode ser representado.

Na computação, o termo foi usado pela primeira vez no fim dos anos 70 e desde então, ontologias têm sido aplicadas em uma infinidade de áreas da ciência da computação, uma das motivações foi a necessidade de criar representações de princípios de conhecimento de domínio na comunidade de compartilhamento e reutilização de conhecimento em IA \cite{guizzardi2005}.

Em áreas relacionadas com Modelagem Conceitual, ontologias são usadas de acordo com dua definição em Filosofia. Na Inteligência Artificial, Engenharia de Software e \textit{Web} Semântica geralmente ontologia é usada como: (i)um artefato concreto de engenharia desenhado com um intuito específico sem focar muito em questões de fundamentação ou (ii) uma representação de um domínio particular que pode ser expressa em alguma linguagem de representação de conhecimento (RDF, OWL, F-Logic) ou de modelagem conceitual (UML, EER) \cite{guizzardi2008}. Neste trabalho, será considerada a definição ii, que é a mais utilizada na área de Inteligência Artificial.

Segundo \cite{gruber1995} uma ontologia é uma especificação formal e explícita de uma conceituação compartilhada. A partir da definição dada por Gruber, \cite{rocha2014} complementa o pensamento afirmando que “formal”, significa que é definido declarativamente para que possa ser compreendido por agentes inteligentes; “explícito”, significa que os elementos e suas restrições estão claramente definidos; “conceituação”, representa um modelo abstrato de um campo de conhecimento ou um universo limitado de discurso; “compartilhado”, indica que é um conhecimento consensual, uma terminologia comum do campo modelado \cite{rocha2014}.

\cite{lv2011} afirma que o principal papel das ontologias é melhorar a comunicação entre humanos ou entre computadores, especificando a semântica do aparato simbólico usado no processo de comunicação.

Ontologias podem ser representadas de maneira formal, usada para que elas possam ser processada por comutadores, ou de maneira gráfica, usada para facilitar a compreensão humana \cite{isotani2015}. As linguagens mais populares para descrever ontologias são RDF, RDF-S e OWL \cite{mcguinness}.


\section{OWL}

A web atual possui deficiências que serão superadas com o uso da Web Semântica. Portanto, se faz necessária uma linguagem para descrever as ontologias que permitem a Web Semântica. A\textit{Web Ontology Language} (OWL) OWL é uma linguagem para definir ontologias e dados individuais associados. Sistemas de raciocínio baseados em quadros, lógicas de descrição e linguagens da Web existentes influenciaram o desenvolvimento da OWL, que foi desenvolvida para representar informações analisáveis por computador \cite{lacy2005}.

A OWL é projetada para ser usada por aplicações que precisam processar o conteúdo da informação em vez de apenas apresentar informações aos seres humanos. OWL fornece uma maior facilidade para a máquina interpretar o conteúdo da Web do que linguagens suportadas por XML, RDF e RDF Schema (RDF-S), pois fornece vocabulário adicional junto com uma semântica formal. Ela possui três sub-linguagens cada vez mais expressivas: OWL Lite, OWL DL e OWL Full \cite{mcguinness}. Na Figura \ref{fig:owlsub} é possível observar uma pequena definição sobre cada uma dessas sub-linguagens de OWL.

\begin{figure}[H]
\centering
\includegraphics[width=.9\textwidth]{Figuras/owl_sub.png}
\caption{Sub-linguagens de OWL}
\label{fig:owlsub}
\end{figure}

Através da definição do conceito de bicicleta, é possível ter um exemplo de uma representação formal: uma bicicleta (\textit{bicycle}) é um tipo de veículo (\textit{vehicle}) de transporte, ou seja, pode ser feita a relação “\textit{bicycle is-a vehicle}”. Além disso, um bicicleta pode ser categorizada em bicicletas esportivas (\textit{sport bicycle}) e bicicletas para uso na cidade (\textit{city bicycle}). Por fim, é possível identificar os atributos e propriedades da bicicleta como cor, peso, tamanho, etc. O resultado disto, pode ser visto na Figura \ref{fig:codowl}, onde se encontra parte do código de uma ontologia em OWL que descreve uma bicicleta \cite{isotani2015}.


\begin{figure}[H]
\centering
\includegraphics[width=.9\textwidth]{Figuras/codigo_owl.png}
\caption{Exemplo de código OWL. Fonte: http://hozo.jp/}
\label{fig:codowl}
\end{figure}

\subsection{Elementos básicos}
 
Para construir uma ontologia OWL, são necessários alguns elementos básicos, são eles as classes, instâncias das classes (indivíduos) e os relacionamentos entre instâncias (propriedades).
 
Classes fornecem um mecanismo de abstração para agrupar recursos com características semelhantes. Cada classe OWL é associada a um conjunto de indivíduos, chamado de extensão de classe. Os indivíduos na extensão de classe são chamados de instâncias da classe \cite{mcguinness}. 

Cada indivíduo na OWL é membro da classe \textit{owl:Thing}. Assim, ela é superclasse de todas as classes OWL definidas pelos usuários. Além disso, existe a classe \textit{owl:Nothing} (não possui instâncias) que é uma subclasse de todas as classes OWL. Uma classe é sintaticamente representada como uma instância nomeada da \textit{owl:Class} \cite{de2005}.

Uma propriedade é uma relação binária que pode ser de dois tipos:

• Propriedades de dados tipados: relação entre indivíduos e valores
de dados;

• Propriedades de objetos: relação entre indivíduos \cite{welty2004}.

\subsubsection{Restrições em propriedade}

OWL permite que sejam impostas restrições sobre propriedades, restrição é um tipo especial de descrição de classe, isto é, descreve uma classe anônima de indivíduos que satisfazem as restrições \cite{de2005}.

A restrição \textit{allValuesFrom} e a \textit{someValuesFrom} são exemplos de restrição de valores. A \textit{owl:allValuesFrom} é usada para descrever uma classe de todos os indivíduos para os quais todos os valores da propriedade em questão são membros da extensão de classe da descrição da classe ou são valores de dados dentro do intervalo de dados especificado \cite{welty2004}, um exemplo do uso dessa propriedade por ser observado na Figura \ref{fig:allvalues}.

\begin{figure}[H]
\centering
\includegraphics[width=.9\textwidth]{Figuras/all_values.PNG}
\caption{Exemplo de uso da restrição allValuesFrom}
\label{fig:allvalues}
\end{figure}

A restrição \textit{someValuesFrom} descreve uma classe de todos os indivíduos para os quais pelo menos um valor da propriedade em questão é uma instância da descrição da classe ou um valor de dados no intervalo de dados \cite{welty2004}. Um exemplo do uso dessa propriedade por ser visto na Figura \ref{fig:somevalues}.

\begin{figure}[H]
\centering
\includegraphics[width=.9\textwidth]{Figuras/some_values.PNG}
\caption{Exemplo de uso da restrição someValuesFrom}
\label{fig:somevalues}
\end{figure}

\subsection{Conjunto de Operadores}

OWL permite combinações booleanas arbitrárias para manipulações das extensões de classes, chamados de conjunto de operadores, eles podem ser vistos como representando os operadores AND, OR e NOT nas classes. Os três operadores obtêm os nomes padrão de operador de conjunto: interseção, união e complemento \cite{bechhofer2004}.

• \textit{owl:intersectionOf}: semelhante ao operador lógico AND, declara uma classe cuja extensão de classe contém somente indivíduos que são membros da extensão de classe de todas classes descritas em uma lista \cite{bechhofer2004}. Um exemplo de um trecho de código que utiliza este operador pode ser encontrado na Figura \ref{fig:opand}.

\begin{figure}[H]
\centering
\includegraphics[width=.9\textwidth]{Figuras/op_and.PNG}
\caption{Operador intersectionOf}
\label{fig:opand}
\end{figure}


• \textit{owl:unionOf}: semelhante ao operador lógico OR, declara uma classe anônima cuja extensão de classe contém indivíduos que ocorrem em pelo menos uma extensão de classe das classes descritas em uma lista \cite{bechhofer2004}. Um exemplo de um trecho de código que utiliza este operador pode ser encontrado na Figura \ref{fig:opor}.

\begin{figure}[H]
\centering
\includegraphics[width=.9\textwidth]{Figuras/op_or.PNG}
\caption{Operador unionOf}
\label{fig:opor}
\end{figure}


• \textit{owl:complementOf}: semelhante ao operador lógico NOT,  declara uma classe cuja extensão de classe contém exatamente os indivíduos que não pertencem à extensão de classe de outra classe, que é o objeto da declaração \cite{bechhofer2004}. Um exemplo de um trecho de código que utiliza este operador pode ser encontrado na Figura \ref{fig:opnot}.

\begin{figure}[H]
\centering
\includegraphics[width=.9\textwidth]{Figuras/op_not.PNG}
\caption{Operador complementOf}
\label{fig:opnot}
\end{figure}


\subsection{Sintaxe Manchester OWL}

A sintaxe \textit{Manchester OWL} \cite{horridge2006} foi criada para suprir a necessidade que algumas sintaxes geravam devido a falta de símbolos não-lógicos, pois, a maioria utiliza símbolos de lógica de descrição. Assim, a \textit{Manchester Sintax} fornecer aos casos em que são necessários utilizar símbolos não lógicos, uma sintaxe que facilita a criação de ontologias. Ela foi projetada principalmente para apresentar e editar expressões de classe em ferramentas, mas também pode ser usada para representar ontologias completas \cite{horridge2006}.

A Manchester OWL é uma sintaxe que de fácil utilização para para descrições OWL \cite{horridge2009}. Ela é influenciada tanto pela sintaxe \textit{Abstract OWL} quanto pela sintaxe \textit{DL style}, que usa símbolos lógicos de descrição, como o quantificador universal ($\forall$) ou o quantificador existencial ($\exists$). A sintaxe  Manchester usa símbolos matemáticos especiais como $\exists$, $\forall$, ¬ para serem substituídos por palavras-chave como “\textit{some}”, “\textit{only}” e “\textit{not}” respectivamente \cite{yauri2012}. Outro ponto diz respeito a palavra-chave \textit{"that"}, que possui a mesma função da palavra-chave \textit{"and"}, porém foi introduzida na sintaxe apenas para fazer com que a leitura das expressões fossem mais fluidas \cite{horridge2006}. Isso pode ser visto na Figura \ref{fig:manchester_exemplo}; onde existe \textit{"that"} pode ser substituído por \textit{"and"}, e vice-versa, e continuará com a mesma semântica.

Desde que o Manchester OWL Syntax foi criado, tornou-se a sintaxe padrão do Protegé OWL \cite{horridge2006}. No caso do Protegé \cite{noy2001}, ele estende a sintaxe para permitir uma apresentação ainda mais compacta em algumas situações (por exemplo, para explicação) \cite{horridge2009}.

Na Figura \ref{fig:manchester_exemplo} é possível ver a definição de uma expressão de classe. Essa expressão descreve um conjunto de pessoas que têm pelo menos um filho, que têm filhos que são apenas homens.

\begin{figure}[H]
\centering
\includegraphics[width=0.5\textwidth]{Figuras/manchester_exemplo.png}
\caption{Exemplo de uma expressão de classe. Fonte: \cite{horridge2006}}
\label{fig:manchester_exemplo}
\end{figure}

\section{Processamento de Linguagem Natural}

O Processamento de Linguagem Natural é uma área de pesquisa e aplicação que explora como os computadores podem ser usados para entender e manipular texto ou discurso em linguagem natural para fazer coisas úteis. Os pesquisadores desta área buscam reunir conhecimentos sobre como os seres humanos entendem e usam a linguagem para que ferramentas e técnicas apropriadas possam ser desenvolvidas para fazer com que os sistemas de computador entendam e manipulem a linguagem natural para realizar as tarefas desejadas \cite{chowdhury2003}.

Em seu trabalho \cite{liddy2001} define PLN como uma gama teoricamente motivada de técnicas computacionais para analisar e representar textos que ocorrem naturalmente em um ou mais níveis de análise linguística com a finalidade de obter processamento de linguagem semelhante ao humano para uma série de tarefas ou aplicações.

\section{Pellet}

A máquina de inferência Pellet, é implementado em Java e é \textit{open source}, ele fornece uma gama de recursos como: suporte a regras, conexão de raciocínio, identificação de axiomas. Para tornar seus recursos de raciocínio acessível aos usuários, ele fornece diferentes interfaces \cite{sirin2007}. 

Os recursos do Pellet são expostos a partir de uma API Java, uma interface de linha de comando ou um formulário da Web. Ele fornece acesso programático às funções de raciocínio por meio de duas interfaces diferentes, uma para o kit de ferramentas do Jena e uma para a biblioteca da API do OWL \cite{parsia2004}. No desenvolvimento da ferramenta em questão, o Pellet foi utilizado a partir de uma API Java e suas funções foram acessadas através da biblioteca do OWL. 

\begin{figure}[H]
\centering
\includegraphics[width=1\textwidth]{Figuras/pellet_inconsistencia.png}
\caption{Código de procura por classes inconsistentes}
\label{fig:pellet_inconsistencia}
\end{figure}

Como é possível ver na Figura \ref{fig:pellet_inconsistencia}, a API fornece métodos que possibilitam fazer busca por inconsistência nos dados, a partir de um dado \textit{reasoner}, que por sua vez é criado a partir de uma ontologia. Ainda com a API do Pellet é possível inferir informação sobre uma ontologia, como é possível ver na Figura \ref{fig:pellet_inferencia}.


\begin{figure}[H]
\centering
\includegraphics[width=1\textwidth]{Figuras/pellet_inferencia.png}
\caption{Exemplo de código inferindo informação sobre a ontologia}
\label{fig:pellet_inferencia}
\end{figure}

\section{Arquitetura Monolítica}

A arquitetura monolítica pode ser equiparada a uma aplicação composta por diversos módulos que são compilados de maneira separada e depois interligados, resultando num único e grande programa executável, onde cada módulo pode interagir de maneira livre. No modelo de arquitetura de camadas, o sistema é dividido em níveis sobrepostos. Cada uma dessas camadas possui um grupo de funções, elas só podem ser utilizadas por camadas que sejam superiores. Essa maneira de estruturação baseada em camadas cria uma hierarquia de níveis de acesso, o que pode ser uma vantagem pois fornece uma proteção paras as camadas mais internas, uma outra vantagem é a de isolar as funções, o que facilita na manutenção. Porém, esse modelo encontra uma desvantagem na questão do desempenho, pois, cada nova camada resulta numa mudança no modo de acesso \cite{machado2004}.


\section{Maven}

O Maven é uma ferramenta de integração de projetos Java utilizada no gerenciamento e automação de construção (build) de projetos, além disso, fornece diversas funcionalidades adicionais através do uso de \textit{plugins} e estimula o emprego de melhores práticas de programação \cite{maven}.

A ferramenta surgiu da dificuldades que muitos grupos de desenvolvimento tinham ao gerenciar bibliotecas em grandes projetos, de maneira dinâmica o Maven baixa a biblioteca Java e os seus \textit{plugins} de diversos repositórios que ficam armazenados em cache local, que pode ser atualizado quando sempre que for preciso \cite{oliveira2016}. Sua configuração é feita através do arquivo pom.xml (Project Object Model) onde são declarados o tipo de empacotamento, suas dependências e os repositórios de onde as dependências serão buscadas, podendo ser tanto repositórios locais ou remotos. \cite{junior2014}

A Figura \ref{fig:pom_xml} mostra um trecho do arquivo pom.xml utilizado na implementação da OrCA IDE. É possível ver definições de compilação, como por exemplo a versão do Java que será utilizado, a classe principal que será executada ao projeto ser iniciado, o tipo de empacotamento, versão e id (pacote) que serão utilizados na publicação desses arquivos para o repositório local do maven, bem como as dependências utilizadas pelo projeto.

\begin{figure}[H]
\centering
\includegraphics[width=1\textwidth]{Figuras/pom_xml.png}
\caption{Trecho de arquivo pom.xml do \textit{OrCA} IDE}
\label{fig:pom_xml}
\end{figure}

\section{Compilador}

De maneira geral, um compilador é um programa que tem como entrada um código de programa numa certa linguagem e produz como saída um código de programa numa outra linguagem, preservando o significado do código durante esse processo \cite{grune2012}.

Um compilador é dividido em fases, cada uma delas representam uma etapa do processo de compilação, elas podem ser desenvolvidas de maneira separada, mesmo que na prática elas funcionem intercaladamente \cite{foleiss2009}. 

Em seu livro, \cite{somerville2003} faz uma definição geral, onde ele afirma que os compiladores de linguagem de programação possuem uma arquitetura genérica, ela pode ser dividida em componentes como os ilustrados na Figura \ref{fig:compiladorl}.

\begin{figure}[H]
\centering
\includegraphics[width=.6\textwidth]{Figuras/compilador.png}
\caption{Fases do compilador}
\label{fig:compiladorl}
\end{figure}

• \textbf{Analisador Léxico}: recebe como entrada o código fonte e sua função é ler os caracteres deste código e categorizar suas sequências em \textit{tokens}, que são unidades lógicas denominadas que em seguida serão utilizadas por outras partes do compilador, como o
analisador sintático \cite{foleiss2009}. Um \textit{token} é uma sequência de caracteres que pode ser tratada como uma unidade na gramática de uma linguagem de programação, que por sua vez classifica os \textit{tokens} em um conjunto finito de tipos de \textit{token} \cite{andrew2002}.

Na fase de análise léxica, o analisador utiliza um fluxo de caracteres e produz um fluxo de nomes, palavras-chave e sinais de pontuação, ele descarta espaço em branco e comentários entre os \textit{tokens}, a presença de espaços em branco e comentários, complicaria indevidamente o analisador \cite{andrew2002}.

Diversas atividades com grande importância são realizadas durante essa análise, as que mais se destacam são: conversão numérica, identificação de palavras reservadas, e criação e manutenção das tabelas de símbolos, que serão muito utilizadas durante o processo de compilação \cite{neto2017};

• \textbf{Tabela de Símbolos}: possui informações sobre os nomes das entidades (variáveis, nomes de classes, nomes de objetos etc.) usadas no texto que está sendo traduzido \cite{somerville2003};

• \textbf{Analisador Sintático}: responsável pela análise sintática, o analisador sintático recebe como entrada uma cadeia de \textit{tokens} que foi gerada pelo analisador léxico, e faz a verificação para confirmar se ela pode ser gerada pela gramática da linguem em questão. Assim, espera-se que o analisador mostre qualquer erro de sintaxe \cite{aho2007}. Para isso, ele faz uso da gramática da linguagem que foi pré definida e então constrói uma árvore de sintaxe \cite{somerville2003};

• \textbf{Árvore Sintática}: é uma estrutura interna utilizada para representar o programa a ser compilado \cite{somerville2003};

• \textbf{Analisador Semântico}: realiza a verificação da correção semântica das instruções, a análise é feita através de regras que são previamente definidas e então disparadas durante a fase análise sintática. De modo similar, regras de geração de código são associadas à estrutura sintática da linguagem. O código gerado implementa as operações especificadas pelo programa na linguagem de saída do compilador \cite{barbosa2009}.

A fase de análise semântica de um compilador conecta definições de variáveis a seus usos, verifica se cada expressão tem um tipo correto e traduz a sintaxe abstrata em uma representação mais simples adequada para gerar código de máquina \cite{andrew2002};

• \textbf{Gerador de Código}: percorre a árvore sintática e sem seguida gera códigos de máquina abstrata \cite{somerville2003};

Cada linguagem de programação possui uma gramática, que consiste em símbolos terminais, símbolos não-terminais, um símbolo inicial e produções. Símbolos terminais são os símbolos básicos a partir dos
quais as cadeias são formadas \cite{ferreira2015}.

\chapter{Arquitetura do Sistema}
\label{chap:sistema}

\section{Introdução}
Com a finalidade de criar ontologias a partir de Processamento de Linguagem Natural (PLN), o sistema OrCA IDE (\textit{Ontology vieweR and CreAtor Integrated Development Environment}) foi desenvolvido. Neste capítulo descrevemos detalhes da arquitetura do sistema e suas funcionalidades, bem com a linguagem natural controlada (LNC) utilizada para criação das ontologias. O intuito, ao definir esta LNC, foi facilitar ao máximo o desenvolvimento das ontologias por parte dos usuários do sistema. Na Figura \ref{fig:diagComponenteGeral} é possível observar o diagrama de componentes do sistema. O componente \textit{Ontology} depende do componente \textit{Compiler}, o componente \textit{Reasoner} depende do componente \textit{Compiler}. Já o componente \textit{Graph Viewer} depende do componente \textit{Ontology}. O componente \textit{Compiler} é o principal componente do sistema, no qual todos os demais componentes possuem dependência direta ou indireta.

\begin{figure}[H]
\centering
\includegraphics[width=.7\textwidth]{Figuras/DiagramadeComponentes.png}
\caption{Diagrama de componentes geral}
\label{fig:diagComponenteGeral}
\end{figure}

\section{Arquitetura}

O sistema foi implementado utilizando a arquitetura monolítica, visando otimizar o desenvolvimento e facilitar a manutenção do código fonte. Qualquer desenvolvedor e/ou programador que visualizar a estrutura do projeto e desejar realizar ajustes, não terá dificuldades em achar seus componentes. Na Figura \ref{fig:pacotesJava} é possível visualizar todos os pacotes do projeto.

\begin{figure}[H]
\centering
\includegraphics[width=.6\textwidth]{Figuras/estrutura.png}
\caption{Pacotes de classes no projeto}
\label{fig:pacotesJava}
\end{figure}

\section{Componentes}
O sistema possui 4 componentes (telas): \textit{Compiler}, \textit{Ontology}, \textit{Reasoner} e \textit{Graph Viewer}. Detalhes de cada componente do sistema serão vistos nas subseções seguintes. Todos os componentes que serão descritos a seguir podem ser localizados no pacote exibido na Figura \ref{fig:pacotesView}. 

\begin{figure}[H]
\centering
\includegraphics[width=.6\textwidth]{Figuras/pacote_view.png}
\caption{Pacote contendo os principais componentes.}
\label{fig:pacotesView}
\end{figure}

\subsection{Componente Compiler}

O \textit{compiler} é responsável por receber a LNC do usuário e retornar se essa LNC pode ou não ser compilada, bem como apresentar eventuais erros que impossibilitam a compilação. A Tela completa pode ser vista na Figura \ref{fig:telaCompiler}.


\begin{figure}[H]
\centering
\includegraphics[width=.8\textwidth]{Figuras/tela_compiler.png}
\caption{Tela \textit{Compiler} completa}
\label{fig:telaCompiler}
\end{figure}

Como apresentado na Figura \ref{fig:telaCompiler1}, o campo de texto chamado \textit{Source Ontology} é onde o usuário irá inserir as sentenças usando a LNC desenvolvida. O botão \textit{Clear} tem a função de apagar todo o conteúdo desse campo de texto, e o botão \textit{COMPILE} faz a compilação das sentenças em ontologias.

\begin{figure}[H]
\centering
\includegraphics[width=.8\textwidth]{Figuras/tela_compiler1.png}
\caption{Tela \textit{Compiler}: parte da edição da Linguagem Natural}
\label{fig:telaCompiler1}
\end{figure}

Na mesma tela, porém no campo Console (Vide Figura \ref{fig:telaCompiler2}), é possível ver a saída do compilador com uma mensagem amigável para o usuário. Há 5 tipos de saídas no console: 

\begin{enumerate}
  \item \textbf{Lexical OK}: passou sem erros pelo analisador léxico;
  \item \textbf{Sintatic OK}: passou sem erros pelo analisador sintático;
  \item \textbf{Semantic OK}: passou sem erros pelo analisador semântico;
  \item \textbf{Conversor OK}: a linguagem utilizada pelo usuário foi convertida com sucesso para uma ontologia.
  \item \textbf{Erros}: ao passo que algum erro seja encontrado, o console (Figura \ref{fig:telaCompiler2}) irá mostrar em qual linha, coluna e palavra aconteceu o erro.
\end{enumerate}

\begin{figure}[H]
\centering
\includegraphics[width=.8\textwidth]{Figuras/tela_compiler2.png}
\caption{Tela \textit{Compiler}: retorno da compilação da linguagem natural}
\label{fig:telaCompiler2}
\end{figure}

\subsection{Componente Ontology}

A \textit{Ontology} mostra a saída da ontologia (em OWL) após a compilação das sentenças dadas pelos usuários. Este campo não é editável, porém, embora o usuário não possa inserir ou remover texto, o modelo de visualização pode ser alterado através do menu suspenso chamado \textit{Output Model} para as seguintes sintaxes: \textit{RDF/XML} (padrão), \textit{KRSS2, Latex, Manchester OWL Syntax, OWL/XML, OWL Functional Syntax} e \textit{Turtle}. Todas essas sintaxes são fornecidas pela \textit{Manchester OWL API}. 

O botão \textit{Graph} tem a função de abrir o componente \textit{Graph}, e o botão \textit{Copy} a de enviar o texto que está aparecendo no campo de texto para a área de transferência. Na Figura \ref{fig:telaOntology} é possível observar a Tela \textit{Ontology} completa.

\begin{figure}[H]
\centering
\includegraphics[width=.8\textwidth]{Figuras/tela_ontology.png}
\caption{Tela Ontology}
\label{fig:telaOntology}
\end{figure}

\subsection{Componente Reasoner}

Se tratando de ontologias, uma máquina de inferência (\textit{reasoner}) é um \textit{software} capaz de inferir consequências lógicas sobre uma ontologia. O campo de texto superior do componente \textit{Reasoner} apresenta a saída (Deduções e Inconsistências) que o \textit{reasoner} integrado a máquina de inferência (\textit{Pellet}) consegue realizar.

O botão \textit{Reason} refaz o processamento de deduzir sobre a ontologia base. O campo de texto \textit{Ontology Inferred} é o código \textit{OWL} da nova ontologia criada a partir dos axiomas deduzidos. O botão \textit{Save on Text File} tem a função de abrir uma caixa de diálogo permitindo que o usuário salve essa nova ontologia em seu computador. Apresentamos na Figura \ref{fig:telaReasoner} a tela completa. Os casos de uso da tela \textit{Ontology} são apresentados na Figura \ref{fig:ucontology}. 

\begin{figure}[H]
\centering
\includegraphics[width=.8\textwidth]{Figuras/tela_reasoner.png}
\caption{Tela Reasoner}
\label{fig:telaReasoner}
\end{figure}

\begin{figure}[H]
\centering
\includegraphics[width=.6\textwidth]{Figuras/UseCaseOntologia.png}
\caption{Caso de Uso da tela \textit{Ontology}} 
\label{fig:ucontology}
\end{figure}

\subsection{Componente Graph Viewer}

O componente \textit{Graph Viewer} é apenas um facilitador gráfico, a fim de mostrar a ontologia compilada em forma de grafo. Apresentamos na Figura \ref{fig:telGraph} a tela \textit{Graph Viewer}, que é responsável por apresentar ao usuário uma perspectiva de grafo da ontologia criada a partir das sentenças em LNC. Nela é possível aumentar e diminuir zoom, assim como exportar tanto em arquivo de imagem no formato SVG como em texto no formato JSON. É possível também movimentar os nós, bem como aumentar ou diminuir o tamanho desses, para facilitar a visualização. 

\begin{figure}[H]
\centering
\includegraphics[width=.7\textwidth]{Figuras/tela_graph.png}
\caption{Tela Graph Viewer}
\label{fig:telGraph}
\end{figure}

Os casos de uso da tela \textit{Graph Viewer} são apresentados na Figura \ref{fig:ucgv}.

\begin{figure}[H]
\centering
\includegraphics[width=.7\textwidth]{Figuras/UseCaseGrafo.png}
\caption{Caso de Uso da tela \textit{Graph Viewer}} 
\label{fig:ucgv}
\end{figure}

\section{Componente Compilador}

Um compilador é um tipo de programa que traduz um código descrito em alto nível, ou seja, mais legível e fácil de entender pelo ser humano, em um código semanticamente equivalente em baixo nível, não legível pelo ser humano e não necessariamente executável pela máquina. Geralmente um compilador não traduz imediatamente um código de máquina, mas sim um código intermediário, em linguagem simbólica, só a partir deste momento o código é traduzido para uma linguagem de máquina \cite{grune2012}. De forma semelhante, o compilador desenvolvido para a OrCA IDE não irá traduzir o código descrito pelo usuário para um código de máquina, mas sim para um código "intermediário", o \textit{OWL 2 DL}, que para os fins desse projeto já é o código final.

Para um programa ser definido como um compilador, ele precisa ter alguns processos, são eles: análise léxica, análise sintática, análise semântica e conversor. Juntos, esses quatro processos formam  o ciclo necessário para um compilador verificar se o código descrito pelo usuário não possui erros e pode ser compilado. O fluxo de um compilador padrão pode ser observado na Figura \ref{fig:compilerscheme}.

\begin{figure}[H]
\centering
\includegraphics[width=.8\textwidth]{Figuras/compiladorscheme.png}
\caption{Fluxo de um compilador}
\label{fig:compilerscheme}
\end{figure}

O fluxo que o compilador OrCA IDE segue é o mesmo de um compilador padrão. Inicialmente o código é examinado pelo Analisador Léxico, caso haja algum erro que esse analisador possa detectar, o processo de compilação é suspenso, e é informado ao usuário, na tela \textit{Compiler} > \textit{Console}, a linha e a posição onde o erro ocorreu. Caso as verificações desse analisador não detectem erro o próximo passo é iniciado, o Analisador Sintático. Neste passo, acontece de igual forma ao Analisador Léxico, porém, ele faz suas próprias verificações. Caso não sejam detectados erros, o passo seguinte é inicializado, o Analisador Semântico. De maneira semelhante aos passos anteriores, o Analisador Semântico faz suas próprias verificações e análises. Se nenhum erro for encontrado, a próxima etapa é iniciada, a conversão. Nesta fase, as sentenças em LNC inseridas pelos usuários, passam por todo o processo apresentado de compilação e o resultado é um código \textit{OWL 2 DL}. Na Figura \ref{fig:uclnc} é possível verificar os casos de uso dessa tela.

\begin{figure}[H]
\centering
\includegraphics[width=.6\textwidth]{Figuras/UseCaseLNC.png}
\caption{Caso de Uso da tela \textit{Compiler}} 
\label{fig:uclnc}
\end{figure}


\subsection{Gramática}

Idiomas como o português, inglês e espanhol possuem um conjunto de regras e normas bem definidas. Isto permite que pessoas com conhecimento destas regras comuniquem-se entre si, tanto no ato de escrever como no de ler. O conjunto dessas regras é chamado gramática. Assim como em um idioma normal, um compilador precisa de regras e normas gramaticais. Isso garante que, ao usuário escrever seu código, ele será entendido por outras pessoas, assim como pelo compilador. Na Tabela \ref{tab:tabelaGramatica} é possível observar a gramática definida para o compilador OrCA IDE. Cada linha é chamada de Regra de Produção. A coluna da esquerda é o título da regra, enquanto a coluna da direita é a regra em si. Para fins de leitura, chamada de regra é feita com <...>, por exemplo <OP>. Chamada de terminal é feita sem <...>, por exemplo \textit{some} ou (. O Símbolo "||" é apenas um separador entre as regras.

\begin{table}[H]
\centering
\caption{Gramática OrCA IDE.}
\label{tab:tabelaGramatica}
\begin{tabular}{|l|l|}
\hline Regra  & Definição  \\ 
\hline <OP>  & \begin{tabular}[l]{@{}l@{}} some || all || only || and || or || isa || equivalent || that \end{tabular} \\
\hline <ID>  & \begin{tabular}[l]{@{}l@{}} a-Z* \end{tabular} \\
\hline <NUM> & \begin{tabular}[l]{@{}l@{}} 0-9* \end{tabular} \\
\hline <FIM> & \begin{tabular}[l]{@{}l@{}} ; || <DEF> || $\varepsilon$ \end{tabular} \\ 
\hline <DEF> & 
\begin{tabular}[l]{@{}l@{}} 
<ID> <OP> <ID> <FIM> \textbf{||} \\
<ID> <OP> (<DEF>) <FIM> || \\
(<DEF>) <OP> <ID> <FIM> || \\
<ID> <OP> <DEF> <FIM> || \\
\end{tabular} \\
\hline
\end{tabular} 
\end{table} 


\section{Linguagem Natural Controlada}

Para que o usuário consiga utilizar o sistema, se faz necessário que ele possua o conhecimento da LNC que será definida a seguir, na Tabela \ref{tab:tabelaPalavrasReservadas} observamos as palavras reservadas, suas respectivas descrições e um exemplo de sua aplicação.

\begin{table}[H]
\centering
\caption{Palavras Reservadas}
\label{tab:tabelaPalavrasReservadas}
\begin{tabular}{|c|l|l|}
\hline Reservado  & Descrição & Exemplo  \\ 
\hline isa & \begin{tabular}[l]{@{}l@{}} Uma classe X é um \\tipo de classe Y.\end{tabular} & Gato isa Felino \\ \hline
equivalent & \begin{tabular}[l]{@{}l@{}} Uma classe X é um tipo \\ de classe irmã de Y.\end{tabular} & JanelaBanheiro equivalent JanelaCasa \\ \hline and / that & \begin{tabular}[l]{@{}l@{}} Uma classe X interseção\\ com uma classe Y.\end{tabular} & (Correr and Pular) isa ExercicioFisico  \\ \hline or & \begin{tabular}[l]{@{}l@{}}Uma classe X disjunção\\ de uma classe Y. \end{tabular} & Trabalhar isa nor(Dormir or Comer) \\ \hline only & \begin{tabular}[l]{@{}l@{}}Uma propriedade X\\ only classe Y. \end{tabular} & podeLatir only Cachorro \\ \hline some & \begin{tabular}[l]{@{}l@{}}Uma propriedade X\\ some classe Y. \end{tabular} & andarNoMuro some Gato \\ \hline all & \begin{tabular}[l]{@{}l@{}}Uma propriedade X \\ all classe Y. \end{tabular} & comerGrama all Vaca \\ \hline not & Negação de uma Classe & not Gato \\ \hline nor & Negação de uma expressão & nor(Gato isa Felino) \\ \hline 
\end{tabular}
\end{table}

Como visto de maneira resumida na Tabela \ref{tab:tabelaPalavrasReservadas}, na Figura \ref{fig:codigoFonteInicial} está o código fonte que o sistema mostra ao ser aberto, para que o usuário entenda o funcionamento da sintaxe.

\begin{figure}[H]
\centering
\includegraphics[width=.8\textwidth]{Figuras/codigo_codigofonte_inicial.png}
\caption{Código inicial}
\label{fig:codigoFonteInicial}
\end{figure}

\subsection{Analisador Léxico}

Este analisador é responsável por fazer o primeiro e menos complexo processamento sobre o código, em LNC, inserido pelo usuário. Esse processamento nada mais é do que receber o texto da interface gráfica e verificar se as palavras e símbolos inseridos pelo usuário são válidos, ou seja, se são aceitos pela gramática do compilador. Essas palavras e símbolos podem ser vistos na Figura \ref{fig:codigoTokenenum}. Com exceção de \textit{END}, \textit{NUMBER} e \textit{IDENTIFIER}, cada palavra e símbolo desta imagem é chamado de "palavra reservada", e não podem ser utilizados para outro fim se não o que foi definido pela gramática deste compilador. O tipo \textit{NUMBER} diz respeito a uma numeração, que pode variar de 1 a ±\textit{n}, ou seja, um número qualquer, já o \textit{IDENTIFIER} diz respeito a uma palavra, que não é reservada pelo compilador, mas foi criada pelo usuário, por exemplo "Cachorro" ou "Vaca". \textit{END} significa que é o final do código inserido pelo usuário. Cada palavra e/ou símbolo inserido pelo usuário é chamado de \textit{token}.

\begin{figure}[H]
\centering
\includegraphics[width=.6\textwidth]{Figuras/codigo_tokenenum.png}
\caption{Palavras reservadas e símbolos permitidos pela gramática}
\label{fig:codigoTokenenum}
\end{figure}

A análise léxica consiste em verificar todos os \textit{tokens}. Caso o \textit{token} não pertença a nenhum dos tipos apresentados na Figura \ref{fig:codigoTokenenum}, irá apresentar uma exceção do tipo \textit{LexicalAnalyzerException}, e o processo de compilação será forçadamente parado (ver Figura \ref{fig:codigoErroLexico}). Caso este \textit{token} pertença a algum dos tipos, ele é adicionado em uma lista de \textit{tokens}. Ao final da análise, essa lista é passada para o Analisador Sintático, encerrando esta análise e mostrando ao usuário que esta etapa foi finalizada com sucesso (ver Figura \ref{fig:codigoSucessoLexico}).

\begin{figure}[H]
\centering
\includegraphics[width=.8\textwidth]{Figuras/codigo_erro_lexico.png}
\caption{Erro do Analisador Léxico sendo mostrado ao usuário na Tela \textit{Compiler}}
\label{fig:codigoErroLexico}
\end{figure}

\begin{figure}[H]
\centering
\includegraphics[width=.3\textwidth]{Figuras/codigo_sucesso_lexico.png}
\caption{Analisador Léxico foi executado com sucesso}
\label{fig:codigoSucessoLexico}
\end{figure}

\subsection{Analisador Sintático}

Após a análise léxica, o analisador sintático é acionado. Ele é responsável por identificar se o código inserido pelo usuário faz parte da gramática que o compilador aceita. Em outras palavras, enquanto o analisador léxico se preocupa em verificar se as palavras são aceitas pelo compilador, o analisador sintático verifica se as palavras foram utilizadas na posição correta. Em alguns casos é utilizado \textit{look ahead} de 1 ou 2 tokens para garantir que a leitura na pilha de execução seja a esperada pelo sistema. A Figura \ref{fig:codigoLookAhead} traz um trecho de código do analisador sintático mostrando a utilização de \textit{look ahead}. Esse código verifica se o item atual da pilha é um \textit{not} e, utilizando 1 \textit{token} de \textit{look ahead}, verifica se o próximo item é um \textit{identifier}, e, utilizando mais 1 \textit{token} de \textit{look ahead}, verifica se esse próximo \textit{token} é um operador (\textit{OR}, \textit{AND}, \textit{ISA}, etc).
 
\begin{figure}[H]
\centering
\includegraphics[width=.8\textwidth]{Figuras/codigo_lookahead.png}
\caption{Trecho de código mostrando a utilização do \textit{look ahead}. }
\label{fig:codigoLookAhead}
\end{figure}

Caso alguma expressão do código inserido pelo usuário não pertença a gramática aceita pelo compilador, uma exceção do tipo \textit{SintaticAnalyzerException} será lançada, como é possível observar na Figura \ref{fig:codigoErroSintatico}. Caso o código do usuário respeite a gramática, esse passo será encerrado, mostrando ao usuário uma mensagem de sucesso como a exibida na Figura \ref{fig:codigoSucessoSintatico}.

\begin{figure}[H]
\centering
\includegraphics[width=.8\textwidth]{Figuras/codigo_erro_sintatico.png}
\caption{Erro do Analisador Sintático sendo mostrado ao usuário na Tela \textit{Compiler}}
\label{fig:codigoErroSintatico}
\end{figure}

\begin{figure}[H]
\centering
\includegraphics[width=.3\textwidth]{Figuras/codigo_sucesso_sintatico.png}
\caption{Analisador Sintático foi executado com sucesso}
\label{fig:codigoSucessoSintatico}
\end{figure}

\subsection{Analisador Semântico}

Posterior a análise sintática esse componente entra em ação. Ele tem um comportamento diferente dos analisadores semânticos de outros compiladores. A função dele é atomizar o código fonte inserido pelo usuário a fim de criar sub expressões de tamanho mínimo, por exemplo "Homem and Mulher", embora essa expressão faça parte de uma expressão maior denominada "(Homem and Mulher) isa Humano". Essa substituição pode ser vista na Figura \ref{fig:codigoAtomizacao} onde se encontra o código do arquivo ControllerSemantic.java que exibe o momento em que a atomização envolvendo essas palavras reservadas é feita. 

\begin{figure}[H]
\centering
\includegraphics[width=.7\textwidth]{Figuras/codigo_atomizacao.png}.
\caption{Código atomização}
\label{fig:codigoAtomizacao}
\end{figure}

Cada expressão atomizada é inserida em uma lista que será enviada para o ControllerConversor. Caso aconteça algum erro na atomização, uma exceção do tipo SemanticAnalyzerException será lançada na view, como é possível observar na Figura \ref{fig:codigoErroSemantico}. Caso o código fonte seja atomizado com sucesso, irá passar para o Controller Conversor.

\begin{figure}[H]
\centering
\includegraphics[width=.7\textwidth]{Figuras/codigo_erro_semantico.png}
\caption{Erro Semântico}
\label{fig:codigoErroSemantico}
\end{figure}

A Figura \ref{fig:codigoErroSemantico} está dividida em duas partes, a parte A mostra quando o código passa com sucesso pelo analisador semântico e a parte B mostra quando houve um erro ao tentar atomizar o código fonte.


\subsection{Conversor}

Esse componente é responsável por converter a linguagem natural inserida pelo usuário em uma ontologia 2 DL. Ao chegar nesse ponto, todas as verificações referentes a análise léxica, sintática e semântica já foram feitas, garantindo que não ocorrerá nenhum tipo de erro no que diz respeito a linguagem natural inserida.

A Figura \ref{fig:codigoConversor} exibe o trecho de código em que ocorre o processo inicial de conversão das expressões AND e THAT, é nessa parte onde é feita a separação quando a expressão é CLASSE vs CLASSE, ou CLASSE vs EXPRESSÃO (vice-versa) ou ainda, EXPRESSÃO vs EXPRESSÃO.

\begin{figure}[H]
\centering
\includegraphics[width=.8\textwidth]{Figuras/codigo_conversor_andthat.png}
\caption{Trecho de código} 
\label{fig:codigoConversor}
\end{figure}

\subsection{Funcionalidades}
O sistema possui algumas funcionalidades básicas no menu contextual (clique direito do mouse sobre o painel) ou na barra de ferramentas (Menu > \textit{Edit}), podendo variar de acordo com o painel, são elas: Copiar, Colar, Cortar, Selecionar tudo, Limpar, Desfazer e Refazer (ver Figura \ref{fig:uclnc}). Na tela \textit{Compiler} o usuário pode inserir sua Linguagem Natural Controlada, bem como compilar, salvar, carregar,  além das opções do menu contextual. 

A tela \textit{Ontology} fornece ao usuário, além das opções do menu contextual: visualizar a ontologia proveniente da sua LNC e mudar formato da ontologia (RDF/XML, Latex, etc), ver Figura \ref{fig:ucontology}.

Na tela \textit{Graph Viewer} é possível aumentar/diminuir zoom, movimentar nós, visualizar informações do nó, exportar o grafo (Json e PNG) e fazer filtros para simplificar o grafo, entre outros. Ver Figura \ref{fig:ucgv}.


\subsection{Considerações Finais}
Concluímos que a ferramenta foi criada com o intuito de facilitar a criação de ontologias por pessoas sem conhecimento em linguagens de programação. Os componentes analisadores - léxico, sintático e semântico - garantem que a linguagem natural inserida seja equivalente a ontologia resultante após a compilação. Podemos perceber também que, pelo fato do usuário poder selecionar o tipo de sintaxe da ontologia, isso possibilita utilizar essa informação para leitura humana (mais fácil de compreender) ou para utilização em outros \textit{softwares} (com informações mais complexas).


\chapter{Experimentos e Resultados}
\label{chap:exp}
Para realização dos experimentos, alguns axiomas foram extraídos da ontologia NTD \cite{ntdo}. %Essa ontologia tem como alvo modelar o domínio de doenças tropicais negligenciadas.%

The Neglected Tropical Diseases (NTDs) are a set of diseases which persist in exactly the physical, psychosocial and economic situation of the poorest, most marginalized populations of the developing world. Most NTDs diseases are chronic diseases which are related to poverty as a promoter of parasitic, bacterial, and also viral or fungal infections. Although of major healthcare impact in developing and threshold countries NTDs are still seen as a rare event in developed countries. 





\chapter{Conclusão}
\label{chap:conclusao}

\section{Trabalhos Futuros}

TEXTO

\bibliographystyle{sbc.bst}
\bibliography{refs}
\addcontentsline{toc}{chapter}{Bibliografia}

\end{document}